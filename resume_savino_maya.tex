%-------------------------------------
% LaTeX Resume
% Author : Maya Savino
%-------------------------------------

\documentclass[letterpaper,12pt]{article}[leftmargin=*]

\usepackage[empty]{fullpage}
\usepackage{enumitem}
\usepackage{ifxetex}
\ifxetex
  \usepackage{fontspec}
  \usepackage[xetex]{hyperref}
\else
  \usepackage[utf8]{inputenc}
  \usepackage[T1]{fontenc}
  \usepackage[pdftex]{hyperref}
\fi
\usepackage{fontawesome}
\usepackage[sfdefault,light]{FiraSans}
\usepackage{anyfontsize}
\usepackage{xcolor}
\usepackage{tabularx}
\usepackage{gensymb}

%-------------------------------------------------- SETTINGS HERE --------------------------------------------------
% Header settings
\def \fullname {Maya Savino}
\def \subtitle {}

\def \phoneicon {\faPhone}
\def \phonetext {+1 (505) 347-9548}

\def \emailicon {\faEnvelope}
\def \emaillink {mailto:maya.savino@proton.me}
\def \emailtext {maya.savino@proton.me}

\def \githubicon {\faGithub}
\def \githublink {https://github.com/mayasavino}
\def \githubtext {/mayasavino}

%\def \websiteicon {\faGlobe}
%\def \websitelink {https://mayasecurity.journoportfolio.com/}
\def \websitetext {\textcolor {white} {
oooooooooooooooooooooo}}

%\def \websitetext { }

\def \headertype {\doublecol} % \singlecol or \doublecol

% Misc settings
\def \entryspacing {-0pt}

\def \bulletstyle {\faAngleRight}

% Define colours
\definecolor{primary}{HTML}{000000}
\definecolor{secondary}{HTML}{0D47A1}
\definecolor{accent}{HTML}{263238}
\definecolor{links}{HTML}{1565C0}

%------------------------------------------------------------------------------------------------------------------- 

% Defines to make listing easier
\def \linkedin {\linkedinicon \hspace{3pt}\href{\linkedinlink}{\linkedintext}}
\def \phone {\phoneicon \hspace{3pt}{ \phonetext}}
\def \email {\emailicon \hspace{3pt}\href{\emaillink}{\emailtext}}
\def \github {\githubicon \hspace{3pt}\href{\githublink}{\githubtext}}
\def \website {\websiteicon \hspace{3pt}\href{\websitelink}{\websitetext}}

% Adjust margins
\addtolength{\oddsidemargin}{-0.55in}
\addtolength{\evensidemargin}{-0.55in}
\addtolength{\textwidth}{1.1in}
\addtolength{\topmargin}{-0.3in}
\addtolength{\textheight}{1.1in}

% Define the link colours
\hypersetup{
    colorlinks=true,
    urlcolor=links,
}

% Set the margin alignment 
\raggedbottom
\raggedright
\setlength{\tabcolsep}{0in}

%-------------------------
% Custom commands

% Sections
\renewcommand{\section}[2]{\vspace{5pt}
  \colorbox{secondary}{\color{white}\raggedbottom\normalsize\textbf{{#1}{\hspace{7pt}#2}}}
}

% Entry start and end, for spacing
\newcommand{\resumeEntryStart}{\begin{itemize}[leftmargin=2.5mm]}
\newcommand{\resumeEntryEnd}{\end{itemize}\vspace{\entryspacing}}

% Itemized list for the bullet points under an entry, if necessary
\newcommand{\resumeItemListStart}{\begin{itemize}[leftmargin=4.5mm]}
\newcommand{\resumeItemListEnd}{\end{itemize}}

% Resume item
\renewcommand{\labelitemii}{\bulletstyle}
\newcommand{\resumeItem}[1]{
  \item\small{
    {#1 \vspace{-2pt}}
  }
}

% Entry with title, subheading, date(s), and location
\newcommand{\resumeEntryTSDL}[4]{
  \vspace{-1pt}\item[]
    \begin{tabularx}{0.97\textwidth}{X@{\hspace{60pt}}r}
      \textbf{\color{primary}#1} & {\firabook\color{accent}\small#2} \\
      \textit{\color{accent}\small#3} & \textit{\color{accent}\small#4} \\
    \end{tabularx}\vspace{-6pt}
}

% Entry with title and date(s)
\newcommand{\resumeEntryTD}[2]{
  \vspace{-1pt}\item[]
    \begin{tabularx}{0.97\textwidth}{X@{\hspace{60pt}}r}
      \textbf{\color{primary}#1} & {\firabook\color{accent}\small#2} \\
    \end{tabularx}\vspace{-6pt}
}

% Entry for special (skills)
\newcommand{\resumeEntryS}[2]{
  \item[]\small{
    \textbf{\color{primary}#1 }{ #2 \vspace{-6pt}}
  }
}

% Double column header
\newcommand{\doublecol}[6]{
  \begin{tabularx}{\textwidth}{Xr}
    {
      \begin{tabular}[c]{l}
        \fontsize{35}{45}\selectfont{\color{primary}{{\textbf{\fullname}}}} \\
        {\textit{\subtitle}} % You could add a subtitle here
      \end{tabular}
    } & {
      \begin{tabular}[c]{l@{\hspace{1.5em}}l}
        {\small#4} & {\small#1} \\
        {\small#5} & {\small#2} \\
        {\small#6} & {\small#3}
      \end{tabular}
    }
  \end{tabularx}
}

% Single column header
\newcommand{\singlecol}[6]{
  \begin{tabularx}{\textwidth}{Xr}
    {
      \begin{tabular}[b]{l}
        \fontsize{35}{45}\selectfont{\color{primary}{{\textbf{\fullname}}}} \\
        {\textit{\subtitle}} % You could add a subtitle here
      \end{tabular}
    } & {
      \begin{tabular}[c]{l}
        {\small#1} \\
        {\small#2} \\
        {\small#3} \\
        {\small#4} \\
        {\small#5} \\
        {\small#6}
      \end{tabular}
    }
  \end{tabularx}
}

\begin{document}
%-------------------------------------------------- BEGIN HERE --------------------------------------------------

%---------------------------------------------------- HEADER ----------------------------------------------------

\headertype{\linkedin}{\github}{\website}{\phone}{\email}{} % Set the order of items here
\vspace{-10pt} % Set a negative value to push the body up, and the opposite

%-------------------------------------------------- EDUCATION --------------------------------------------------
\section{\faGraduationCap}{Education}

  \resumeEntryStart
    \resumeEntryTSDL
      {The New Mexico Institute of Mining and Technology}{2020 -- Present}
      {B.S. Computer Science (in progress part-time): 3.2 GPA}{Socorro, NM}
  \resumeEntryEnd


%-------------------------------------------------- EXPERIENCE --------------------------------------------------
\section{\faPieChart}{Experience}

  \resumeEntryStart
    \resumeEntryTSDL
      {Computer Science Department of NMT}{2022 -- Present}
      {System Administrator}{Socorro, NM}
    \resumeItemListStart
        \resumeItem {Along with maintaining 100+ virtualized servers (hosted with ProxMox on three physical servers), I created a new DNS server for our external network with bind9 on Debian --- replacing our outdated one --- and managed DNS records for our internal network with Azure DNS. I built, configured, and maintained 30+ lab computers (running Kubuntu), including connecting new or recently repaired ones to our network with an Ansible script. Additionally, I fixed six non-functional donated laptops to be used as loaners for students (equipped with Linux Mint), and created a system for lending and keeping track of them. I also created and managed user accounts with our Microsoft Active Directory Service, created and updated documentation, and provided IT support to students and faculty.}
    \resumeItemListEnd
  \resumeEntryEnd

  \resumeEntryStart
    \resumeEntryTSDL
      {Cybersecurity Department of NMT}{2022 -- Present}
      {Lead Developer for CTF (Capture the Flag) Cybersecurity Competitions}{Socorro, NM}
    \resumeItemListStart
      \resumeItem {I created, ran, and moderated competitions based on a variety of cybersecurity concepts. The memory forensics section required participants to analyze a snapshot of a computer's RAM the moment malware was deployed.  The cryptography section contained challenges related to hashing algorithms, ciphers, various methods of character encoding, and barcodes. I also had a section dedicated to various methods of steganographically encoding files or text in images and audio files. My biggest project was the section on memory forensics (detailed below).}
    \resumeItemListEnd
  \resumeEntryEnd

    \resumeEntryStart
    \resumeEntryTSDL
      {Freelance Contracting}{2017 -- 2021}
      {Software QA Testing}{Remote}
    \resumeItemListStart
      \resumeItem {For four years, I was an individual contractor providing software quality assurance testing. One of my biggest repeat clients was leading the development of a virtual management-training card game exercise. I would recruit others and create teams, conducting group testing sessions and gathering the data from all participants, and pay them for their time. I would also test the software on my own using multiple machines. Along with performing compatibility testing, I would test the mechanics of the card game itself, points logic, animations, and video/audio capabilities, as well as any new features that were added. I also had other contracts testing various websites, applications, and other games.}
    \resumeItemListEnd
  \resumeEntryEnd


%-------------------------------------------------- PROJECTS --------------------------------------------------
\section{\faFlask}{Competitions}

  \resumeEntryStart
    \resumeEntryTSDL
      {CAHSI Hackathon}{October 2021}{Great Minds in STEM}{Remote}
    \resumeItemListStart
      \resumeItem {Won 1st place in the Hidden Messages category out of 50 teams.}
    \resumeItemListEnd
  \resumeEntryEnd

  \resumeEntryStart
    \resumeEntryTSDL
      {Tracer FIRE}{February 2021}{Sandia National Laboratories}{Socorro, NM}
    \resumeItemListStart
      \resumeItem {Won 3rd place out of 14 teams.}
    \resumeItemListEnd
  \resumeEntryEnd 
  
{\textcolor {white} {
o\\o\\o\\}}


%-------------------------------------------------- PROGRAMMING SKILLS --------------------------------------------------
\newpage
{\textcolor {white} {
o\\}}

\section{\faGears}{Skills \& Projects}
 \resumeEntryStart
  \resumeEntryS{Programming Languages } {Python, BASH, Javascript, C, HTML, CSS, LaTeX.}
  \resumeEntryS{Familiar Tools } {Nmap, Volatility (digital forensics), Wireshark.}
 \resumeEntryEnd


\resumeEntryStart
    \resumeEntryTSDL
      {Memory Forensics}{}{CTF Competition Section}{}
    \resumeItemListStart
      \resumeItem {The VMs I created for this project included a workstation running Windows 10 and database server with a database I created with sqlite3 running Linux Mint. I set up an internal network and connected both VMs, then created a custom virus with python and bash and deployed the virus, which was a type of ransomware that stole crucial information from the database, then encrypted the database with AES256 and a pseudorandom key based on the time of execution, and finally copied back the information and key to the executor and deleted the original, unencrypted database before deleting itself.}
    \resumeItemListEnd
  \resumeEntryEnd

  \resumeEntryStart
    \resumeEntryTSDL
      {Fengaribot Embedded Software}{}{NMT Rolling Miners II for NASA MINDS}{}
    \resumeItemListStart
      \resumeItem {As part of a research and design competition for NASA MINDS (MUREP Innovative New Designs for Space), I worked with a team to design and assemble a prototype for a lunar rover called the Fengaribot. My responsibility was coding. A Raspberry Pi controlled the 360\degree servos that the legs were attached to, the servo with the spindle that allowed the robot to curl up like a pillbug, and the LIDAR. I used python to create software that enabled the rover to move, curl, and see with the sensor.}
    \resumeItemListEnd
  \resumeEntryEnd

  \resumeEntryStart
    \resumeEntryTSDL
      {Minicomputers Quiz}{}{Personal -- Tailwind CSS and Vue.js}{}
    \resumeItemListStart
      \resumeItem {I created a browser-based quiz on minicomputers in the 1960s and '70s with the Vue.js Javascript framework, HTML, and the Tailwind framework for CSS. I also added instructions on running and editing the files, including tips for running it on Windows.}
    \resumeItemListEnd
  \resumeEntryEnd

  \resumeEntryStart
    \resumeEntryTSDL
      {Pseudo-3D FPS}{}{Personal -- Python, pygame}{}
    \resumeItemListStart
      \resumeItem {A pseudo-3D game is a game in which movement is restricted to a 2D plane, but the environment appears 3D. The player can look side to side, but not up or down, like in the game DOOM. The rays are shot horizontally in an arc from the player view to the other edge, and project a wall slice when they encounter a wall. The size of the wall is determined by the distance from the player; a further distance means a smaller slice, giving the illusion of depth. This project is still a work in process; I've built the map, the raycaster to render the simulated 3D environment from the 2D map, the player, and a good chunk of the main logic.}
    \resumeItemListEnd
  \resumeEntryEnd

\resumeEntryStart
    \resumeEntryTSDL
      {Conway's Game of Life}{}{Personal -- C with SDL}{}
    \resumeItemListStart
      \resumeItem {This project is a C implementation of Conway's Game of Life. The cellular automaton is a zero-player game where the evolution is determined by the initial state of the simulation. Each cell can be alive or dead, and its survival is dependent on the number of nearby living cells. I used the SDL library for graphics, and I provided extensive documentation with Doxygen. For debugging and checking my memory management, I used Valgrind.}
    \resumeItemListEnd
  \resumeEntryEnd


\end{document}
